\chapter{Introduction}\label{sec:introduction}

This is a non-normative chapter covering the basic concepts that govern development and maintenance of
the specification.

\section{Overview}

The UAVCAN/CAN physical layer specification (UCANPHY) defines electromechanical conventions for {UAVCAN/CAN}
optimized for use in the avionics of manned and unmanned aircraft as well as in high-integrity robotic systems.
The goal of UCANPHY is to maximize cross-vendor compatibility, ensure consistency across the ecosystem, and
prevent some common design pitfalls.

The development and maintenance of the UCANPHY specification is carried out by the members of the UAVCAN Consortium.
Information on the Consortium is available via the official website at \href{http://uavcan.org}{uavcan.org}.

Information on UAVCAN and its CAN-based transport named UAVCAN/CAN is published in a separate document
available from the official website.

\section{Document conventions}

Non-normative text, examples, recommendations, elaborations, and other optional items
are contained in footnotes\footnote{This is a footnote.} or highlighted sections as shown below.

\begin{remark}
    Non-normative sections such as examples are enclosed in shaded boxes like this.
\end{remark}

Throughout the document, ``CAN'' implies both Classic CAN and CAN FD, unless specifically noted otherwise.

\section{Management and conformance}

The UAVCAN Consortium is tasked with maintaining and advancing this specification,
as well as performing voluntary conformance testing.
The policies that govern these activities are published on the official website.

Products whose conformance with this specification has been confirmed by the Consortium
are entitled to feature the official \emph{UAVCAN Conformity Mark}
subject to the policies published on the official website.

\section{Referenced sources}

The specification contains references to the following sources:

% Please keep the list sorted alphabetically.
\begin{itemize}
    \item CiA 103 --- Intrinsically safe capable physical layer.
    \item CiA 801 --- Application note --- Automatic bit rate detection.

    \item IETF RFC2119 --- Key words for use in RFCs to Indicate Requirement Levels.

    \item ISO 11898-1 --- Controller area network (CAN) --- Part 1: Data link layer and physical signaling.
    \item ISO 11898-2 --- Controller area network (CAN) --- Part 2: High-speed medium access unit.
\end{itemize}

\section{Revision history}

\subsection{v1.0 -- work in progress}

This is the initial version of the document.
The discussions that shaped the initial version are available to the members of the UAVCAN Consortium.
