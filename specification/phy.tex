\chapter{CAN bus physical layer}\label{sec:phy}

As can be seen from its specification, UAVCAN is mostly agnostic of the parameters of the physical layer.
Despite that, vendors should follow the recommendations provided in this section
in the interest of maximizing the cross-vendor compatibility.

\section{Classic CAN}

Table~\ref{table:phy_parameters_classic_can} lists the recommended parameters of the
ISO 11898-2 Classic CAN physical layer.
The estimated bus length limits are based on the assumption that the propagation delay does not exceed 5 ns/m,
not including additional delay times of CAN transceivers and other components.

\begin{UAVCANSimpleTable}[wide]{ISO 11898-2 Classic CAN physical layer parameters}{|l| X[c] X[c] X[c] X[c] |l|}%
    \label{table:phy_parameters_classic_can}%
    Parameter                           &           \multicolumn{4}{c|}{Value}          & Unit      \\
    Bit rate                            &   1000    &   500     &   250     &   125     & kbit/s    \\
    Permitted sample point location     &   75--90  &   85--90  &   85--90  &   85--90  & \%        \\
    Recommended sample point location   &   87.5    &   87.5    &   87.5    &   87.5    & \%        \\
    Maximum bus length                  &   40      &   100     &   250     &   500     & m         \\
    Maximum stub length                 &   0.3     &   0.3     &   0.3     &   0.3     & m         \\
\end{UAVCANSimpleTable}

Designers are encouraged to implement CAN auto bit rate detection when applicable.
Refer to the CiA 801 application note for the recommended practices.

\begin{remark}
    UAVCAN allows the use of a simple bit time measuring approach,
    as it is guaranteed that any functioning UAVCAN network will always exchange node status messages,
    which can be expected to be published at a rate no lower than 1 Hz,
    and that contain a suitable alternating bit pattern in the CAN ID field.
    Refer to the UAVCAN Specification for details.
\end{remark}

\section{CAN FD}

This section is under development and will be populated in a later revision of the document.

\begin{table}[H]
    \caption{ISO 11898-2 CAN FD physical layer parameters}
    \NoLeftSkip
    \begin{tabu} to \textwidth {|l l| X[c] X[c] X[c] X[c] |l|}
        \hline\rowfont{\bfseries{}}
        \label{table:phy_parameters_can_fd}%
        Parameter                           & Segment       & \multicolumn{4}{c|}{Value}& Unit \\\hline

        \multirow{2}{*}{Bit rate}           & Arbitration   & 1000 & 500  & 250  & 125  & \multirow{2}{*}{kbit/s} \\
                                            & Data          & 4000 & 2000 & 1000 & 500  &                       \\\hline
        \multirow{2}{*}{Permitted SPL}      & Arbitration   & TBD  & TBD  & TBD  & TBD  & \multirow{2}{*}{\%}   \\
                                            & Data          & TBD  & TBD  & TBD  & TBD  &                       \\\hline
        \multirow{2}{*}{Recommended SPL}    & Arbitration   & TBD  & TBD  & TBD  & TBD  & \multirow{2}{*}{\%}   \\
                                            & Data          & TBD  & TBD  & TBD  & TBD  &                       \\\hline
        \multicolumn{2}{|l|}{Maximum bus length}            & TBD  & TBD  & TBD  & TBD  & \multirow{2}{*}{m}    \\
        \multicolumn{2}{|l|}{Maximum stub length}           & TBD  & TBD  & TBD  & TBD  &                       \\\hline

    \end{tabu}
\end{table}
